\nonstopmode{}
\documentclass[a4paper]{book}
\usepackage[times,inconsolata,hyper]{Rd}
\usepackage{makeidx}
\usepackage[utf8]{inputenc} % @SET ENCODING@
% \usepackage{graphicx} % @USE GRAPHICX@
\makeindex{}
\begin{document}
\chapter*{}
\begin{center}
{\textbf{\huge Package `rlefer'}}
\par\bigskip{\large \today}
\end{center}
\ifthenelse{\boolean{Rd@use@hyper}}{\hypersetup{pdftitle = {rlefer: Draw evenly-spaced and non-overlapping curves in a flow field}}}{}
\begin{description}
\raggedright{}
\item[Type]\AsIs{Package}
\item[Title]\AsIs{Draw evenly-spaced and non-overlapping curves in a flow field}
\item[Version]\AsIs{1.0}
\item[Date]\AsIs{2024-03-05}
\item[Author]\AsIs{Pedro Duarte Faria}
\item[Maintainer]\AsIs{Pedro Duarte Faria }\email{pedropark99@gmail.com}\AsIs{}
\item[Description]\AsIs{R package to draw evenly-spaced and non-overlapping curves in a flow field, using the Jobard and Lefer (1997) algorithm.}
\item[License]\AsIs{MIT + file LICENSE}
\item[Imports]\AsIs{dplyr, Rcpp (>= 1.0.12), rlang, tibble}
\item[LinkingTo]\AsIs{Rcpp}
\item[RoxygenNote]\AsIs{7.2.3}
\item[Archs]\AsIs{x64}
\end{description}
\Rdcontents{\R{} topics documented:}
\inputencoding{utf8}
\HeaderA{even\_spaced\_curves}{Draws multiple evenly-spaced and non-overlapping curves in a flow field.}{even.Rul.spaced.Rul.curves}
%
\begin{Description}
Draws multiple evenly-spaced and non-overlapping curves in a flow field,
using the Jobard and Lefer (1997) algorithm.
\end{Description}
%
\begin{Usage}
\begin{verbatim}
even_spaced_curves(
  x_start,
  y_start,
  n_curves,
  n_steps,
  min_steps_allowed,
  step_length,
  d_sep,
  flow_field
)
\end{verbatim}
\end{Usage}
%
\begin{Arguments}
\begin{ldescription}
\item[\code{x\_start}] the x coordinate of the starting point from which the function will start to draw your curve.

\item[\code{y\_start}] the y coordinate of the starting point from which the function will start to draw your curve.

\item[\code{n\_curves}] the number of curves you want to draw.

\item[\code{n\_steps}] the number of steps used to draw each curve.

\item[\code{min\_steps\_allowed}] the minimum number of steps allowed in each curve (see Details for more info).

\item[\code{step\_length}] the length/distance taken in each step.

\item[\code{d\_sep}] the "separation distance", i.e., the amount of distance that each curve must be from neighbouring curves.

\item[\code{flow\_field}] a 2D matrix with double values, each double value represents an angle value.
\end{ldescription}
\end{Arguments}
%
\begin{Details}
You can use this function to draw multiple curves in a flow field.
Each curve will be non-overlapping and evenly-space between it's
neighbors.

In essence, this function takes a single starting point (\code{x\_start} and \code{y\_start}) in the flow field,
and it starts to draw a initial curve in the flow field. After that, the function starts a loop process,
to draw \code{n\_curves - 1} curves from this initial curve. In other words, all the curves that are drawn
into the flow field are derived from this initial curve.

In each step of the way, the function will check if the current curve that is being
drawn is too close to it's, by calculating it's distance to the existing curves
around it. If the current curve is getting too close to a neighbor curve, then,
the function will stop drawing the current curve, and will start to draw the
next curve in the queue.

If the function starts to draw a new curve, but the starting point of this new curve is already
too close to other existing curves, then, the function completely drops this curve (i.e.
it "gives up" on drawing this curve), and jumps to the next curve in the queue.

Also, if the function draws a new curve, but this curve have less than
\code{min\_allowed\_steps} steps, then, this curve is also completely dropped.
This avoids getting a high number of curves that are too short.

In other words, it is not guaranteed that this function will draw exactly \code{n\_curves} curves
into the field, because, it might not have enough space for \code{n\_curves} curves, considering your current settings.
So, the function
will attempt to draw as many curves as possible. As long as they are not overlapping
each other, and they are not too close to other neighbouring curves, the function will
continue to draw curves into the field.

For more details about how the algorithm works, check: \url{https://pedro-faria.netlify.app/posts/2024/2024-02-19-flow-even/en/}
\end{Details}
%
\begin{Value}
This function returns a \code{tibble} object with 6 columns:
\begin{itemize}

\item{} \code{curve\_id}: the ID of the curve.
\item{} \code{x}: the x coordinates of each point that represents the curve.
\item{} \code{y}: the y coordinates of each point that represents the curve.
\item{} \code{direction\_id}: which direction that the algorithm was following when drawing the current point (0 means from left to right, 1 means from right to left).
\item{} \code{step\_id}: the ID (or the number) of the current step.
\item{} \code{steps\_taken}: the number of steps taken to draw the current curve.

\end{itemize}

\end{Value}
%
\begin{References}
Jobard, Bruno, and Wilfrid Lefer. 1997. “Creating Evenly-Spaced Streamlines of Arbitrary Density.” In Visualization in Scientific Computing ’97, edited by Wilfrid Lefer and Michel Grave, 43–55. Vienna: Springer Vienna.
\end{References}
%
\begin{Examples}
\begin{ExampleCode}
library(ambient)
set.seed(50)
flow_field <- noise_perlin(c(240, 240))
# The coordinates x = 45 and y = 24 are used as the starting point:
curves <- even_spaced_curves(
  45, 24,
  100,
  30,
  5,
  0.01*240,
  0.5,
  flow_field
)

\end{ExampleCode}
\end{Examples}
\inputencoding{utf8}
\HeaderA{non\_overlapping\_curves}{Draws multiple non-overlapping curves in a flow field.}{non.Rul.overlapping.Rul.curves}
%
\begin{Description}
Draws multiple non-overlapping curves in a flow field,
using the Jobard and Lefer (1997) algorithm.
\end{Description}
%
\begin{Usage}
\begin{verbatim}
non_overlapping_curves(
  starting_points,
  n_steps,
  min_steps_allowed,
  step_length,
  d_sep,
  flow_field
)
\end{verbatim}
\end{Usage}
%
\begin{Arguments}
\begin{ldescription}
\item[\code{starting\_points}] a list object, with the x and y coordinates of the starting points for each curve.

\item[\code{n\_steps}] the number of steps used to draw each curve.

\item[\code{min\_steps\_allowed}] the minimum number of steps allowed in each curve (see Details for more info).

\item[\code{step\_length}] the length/distance taken in each step.

\item[\code{d\_sep}] the "separation distance", i.e., the amount of distance that each curve must be from neighbouring curves.

\item[\code{flow\_field}] a 2D matrix with double values, each double value represents an angle value.
\end{ldescription}
\end{Arguments}
%
\begin{Value}
This function returns a \code{tibble} object with 6 columns:
\begin{itemize}

\item{} \code{curve\_id}: the ID of the curve.
\item{} \code{x}: the x coordinates of each point that represents the curve.
\item{} \code{y}: the y coordinates of each point that represents the curve.
\item{} \code{direction\_id}: which direction that the algorithm was following when drawing the current point (0 means from left to right, 1 means from right to left).
\item{} \code{step\_id}: the ID (or the number) of the current step.
\item{} \code{steps\_taken}: the number of steps taken to draw the current curve.

\end{itemize}

\end{Value}
%
\begin{References}
Jobard, Bruno, and Wilfrid Lefer. 1997. “Creating Evenly-Spaced Streamlines of Arbitrary Density.” In Visualization in Scientific Computing ’97, edited by Wilfrid Lefer and Michel Grave, 43–55. Vienna: Springer Vienna.
\end{References}
%
\begin{Examples}
\begin{ExampleCode}
library(ambient)
set.seed(50)
flow_field <- noise_perlin(c(240, 240))

\end{ExampleCode}
\end{Examples}
\printindex{}
\end{document}
